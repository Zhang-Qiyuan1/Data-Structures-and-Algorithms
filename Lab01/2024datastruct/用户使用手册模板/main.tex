% --------------------------------------------------------------
% This is all preamble stuff that you don't have to worry about.
% Head down to where it says "Start here"
% --------------------------------------------------------------
\documentclass[12pt]{article}

\usepackage[margin=1in]{geometry} 
\usepackage{amsmath,amsthm,amssymb}
\usepackage[UTF8]{ctex}  % 用于中文
\usepackage{graphicx}  % 添加这行来加载插图功能

\usepackage{xcolor}
\usepackage{listings}
\lstset{basicstyle=\ttfamily,
  showstringspaces=false,
  commentstyle=\color{red},
  keywordstyle=\color{blue}
}

\linespread{1.0}  % 设置1倍行间距
\usepackage{geometry}
\geometry{
    left=2cm,
    right=2cm,
    top=3cm,
    bottom=3cm,
}
\usepackage{titling}
\usepackage[pagebackref]{hyperref} % 引用
\usepackage{cleveref}
\crefname{figure}{Fig}{Figs}

\begin{document}
\renewcommand{\labelitemii}{$\circ$} % 第二层级符号为圆圈

% --------------------------------------------------------------
%                         Start here
% --------------------------------------------------------------

\setlength{\droptitle}{-3cm}
\title{基础实验:用户使用手册(模板)}
\author{姓名:xxx \and 学号:xxx}
\date{2024年xx月xx日}
\maketitle




\section{本软件运行环境}

写明软件程序所需运行环境。


\section{操作指令及预期结果}

\subsection{启动}

如何启动本软件?

默认可执行程序文件在所提交文件夹的software目录中。

%明确指出可执行文件(如.exe可执行程序)等,需写明文件或命令行输入。

\begin{lstlisting}[language=bash,caption={Launch the software in the terminal}]
#!/bin/bash
cd dir_of_your_submission/software 
./your_exe_name [input_filename] [extra_params]
\end{lstlisting}


\subsection{功能与提示}

写明关键操作功能和对应输出提示等。

\subsection{退出}

写明退出操作。

\subsection{注意事项}

使用本软件应注意的事项。

 
% --------------------------------------------------------------
%     You don't have to mess with anything below this line.
% --------------------------------------------------------------

\end{document}
